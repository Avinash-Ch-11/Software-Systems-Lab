\documentclass{article} 
\title{CS-213 : Lab 4 : PHP1 part 1 statement}
\author{Ramchandra Phawade} 

\begin{document}
\maketitle 

\noindent 
\begin{center} 
\noindent
In this lab you will learn to use PHP \\
\noindent
Deadline : 17:30 hours  27 August 2019.
\end{center} 

\hrule
\vspace*{0.5cm}
\noindent 
\emph{Submission guidelines}:\\
 
\noindent 
Please create a directory called ``rollno-lab4-part1" where ``rollno" is your
roll number.
 
\noindent 
Keep all your scripts, html pages and all relevant files in this directory 

\noindent 
Use following command to create a tar ball :
\begin{center}
    tar~-czvvf~rollno-lab4-part1.tgz~~rollno-lab4-part1/ 
\end{center}
 
\noindent 
Upload this tarball ``rollno-lab4-part1.tgz" on moodle. 
 
\noindent 
Do not change cases, and do not deviate from the  
naming scheme for your scripts, directory or tar file to be uploaded. 
 
\hrule
\noindent
\begin{enumerate}  
\item Finding $k$-th smallest number:\\
      Take some integers from the command line as input. \\
      First element is parameter $k$.\\
      Print the $k$-th smallest number. \\

      Validate your input: For example if we feed strings instead of numbers your program should 
      print a custom message, and in which argument it has occurred. \\
 
      (A file called e2.php is given where such error handling is shown, in addition to an example given in the demos.)\\

      (A file called argv.php has example showing how to access command line arguments in PHP). \\
   

     \emph{(name of your file should be q1.php)}. 
 
%\item Word frequency counter\\
%      Take some words as input from the command line. 
%      Print the frequency count (number of occurrences) of each word sorted in alphabetical order. \\
%
%      Frequency count must be case-insensitive. 
%      (e.g. both 'Hey' and 'hey' would be counted under the same word 'hey')
%
%\item Substring Frequency counter\\
%      Take two words an input from the command line. 
%      Let us call first word as $x$ and second word as $y$.\\
%      Count how many times $y$ occurs as a substring in $x$, and the positions in $x$ where it occurs it. 
%
%      For example : \\
%      $x= "abacabab"$ and $y="ab"$
%      then your output should be as follows:\\
%      $3$\\
%      $0$\\
%      $4$\\
%      $6$.
% 
\end{enumerate} 
\end{document}    

